
\documentclass[journal,onecolumn]{IEEEtran}
\usepackage[super]{nth}
\newcommand\tab[1][0.5cm]{\hspace*{#1}}

\ifCLASSINFOpdf
 
\else
 
\fi
\hyphenation{op-tical net-works semi-conduc-tor}


\begin{document}

\title{Project 1:Getting Acquainted}

\author{Sangyeon Lee,
        Mingyang Li,
        and Yuan-hao Cheng\\
        CS 544,
        Spring 2018\\
        April 15, 2018}% <-this % stops a space

\maketitle

% As a general rule, do not put math, special symbols or citations
% in the abstract or keywords.
\begin{abstract}
~~~~This report concerns how to build one’s own kernel and run it on QEMU on a linux server. It consists of 3 sections. The first is about the commands having been performed for the task in chronological order. The next part is on the anatomy of a qemu command, and the last one is on the log of this task.
\end{abstract}



\newpage


\IEEEpeerreviewmaketitle



\section{Log of Commands performed for the Requested Actions}

\begin{itemize}
\item cd /						
								\\(Move to the root)
\item cd scratch/spring2018/	
								\\(Move to the assigned directory for a group project)
\item mkdir Group08				
								\\(Make a group directory)
\item cd Group08				
								\\(Get in the group folder)
\item cp /scratch/bin/acl\_open /scratch/spring2018/Group08/	
								\\(Copy acl\_open file to the group folder) 
\item acl\_open . "[userID"]		
								\\(Give permission to access the group folder only to the members of the group)
\item git clone git://git.yoctoproject.org/linux-yocto-3.19 
								\\(Import Linux source files from the given Git Repository)
\item cd linux-yocto-3.19/		
								\\(Move to a directory that contains a local copy of Linux source files)
\item git checkout tags/v3.19.2	
								\\(Switch to the “v3.19.2” tag)
\item make tags					
								\\(Make tags for the ease of editing codes in the future)
\item cd ..						
								\\(Move to the parent folder, which is the group folder)
\item cp /scratch/files/bzImage-qemux86.bin ./
								\\(Copy a pre-made kernel file to the group folder in order to test on VM machine)
\item cp /scratch/files/core-image-lsb-sdk-qemux86.ext4 ./
								\\(Copy all the necessary files, including the subsequent 3 commands, to the group folder)
\item cp /scratch/opt/environment-setup-i586-poky-linux ./
\item cp /scratch/files/config-3.19.2-yocto-standard ./
\item cp config-3.19.2-yocto-standard ./linux-yocto-3.19/.config
\item vi ./linux-yocto-3.19/.config cd linux-yocto-3.19		
								\\(Change the value of CONFIG\_LOCALVERSION="-group08" through vi editor)
\item source environment-setup-i586-poky-linux
								\\(source the necessary environment variables)
\item cd linux-yocto-3.19
								\\(Move to a directory that contains a local copy of Linux source files)
\item make -j4		
								\\(Build a kernel for the group project)
\item cd ..
								\\(Move to the parent folder, which is the group folder)
\item qemu-system-i386 -gdb tcp::5508 -S -nographic -kernel ./linux-yocto-3.19/arch/x86/boot/bzImage -drive file=core-image-lsb-sdk-qemux86.ext4,if=virtio -enable-kvm -net none -usb -localtime --no-reboot --append "root=/dev/vda rw console=ttyS0 debug"
								\\(Run QEMU emulator)
\newline
\newline
\textbf{In another command console}
\newline
\item cd /scratch/spring2018/Group08/linux-yocto-3.19/
								\\(Move to a directory that contains a local copy of Linux source files)
\item gdb vmlinux
								\\(Run gdb debugger with a built Linux image file)
\item (gdb) target remote :5508
								\\(Enter the port number, which has been given when running QEMU the above, for remote debugging)
\item (gdb) continue
								\\(Boot the built Linux kernel)
\end{itemize}



\newpage
\section{Explanation of every flag in the listed qemu command-line}
\begin{itemize}
\item qemu-system-i386
\\: This is an execution file of QEMU for i386 architecture
\item -gdb tcp::5508
\\: This parameter allows to wait for gdb connection on device through  TCP protocol with a port number 5508
\item -S
\\: This parameter allows not to start cpu at start up.
\item -nographic
\\: This parameter means to exclude graphical output, so that a user is provided only with a simple command-line-based Linux.
\item -kernel ./linux-yocto-3.19/arch/x86/boot/bzImage
\\: This parameter allows to utilize a kernel image built for the group project
\item -drive file=core-image-lsb-sdk-qemux86.ext4,if=virtio
\\: This parameter allows to define a new drive to be utilized. Especially, the type of the interface, on which the defined drive is connected, is limited to "virtio"
\item -enable-kvm
\\: This parameter enables KVM full virtualization support
\item -net none
\\: This parameter signifies that no network device should be configured.
\item -usb
\\: This parameter enables the usb driver.
\item -localtime
\\: This parameter allows RTC to start at the current local time.
\item --no-reboot
\\:This parameter allows a machine to exit instead of rebooting
\item -append "root=/dev/vda rw console=ttyS0 debug"
\\:This parameter allows to give the kernel command line arguments.
\end{itemize}

\section{Work Log}
\begin{itemize}
\item \textbf{April \nth{9}, 2018:}
		\\: Start to research on building and running a kernel on a server
\item  \textbf{April \nth{10}, 2018:} 
       \\: Finish building and running a kernel on OS2 server
\item  \textbf{April \nth{11}, 2018:} 
       \\: Allocate the work for LaTex document.
\item \textbf{April \nth{13}, 2018:}
      \\: Combine everyone's LaTex work.
\end{itemize}

                

                




\end{document}
